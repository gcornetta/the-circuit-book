\chapter{Advanced Mathematics}

\begin{summary}
This annex covers more advanced mathematics used in electronics and circuit theory. More specifically, we will deal with differentiation, integration and transformation. In electronics, differentiation is a mathematical process that gives information about the rate of change of physical magnitude (voltage, current, charge, etc.).
\end{summary}

\section{Basic derivatives}
In the sequel, some of the differentials commonly encountered in electronics are listed.

\begingroup
\allowdisplaybreaks
\begin{flalign}
f(x) &= kx^n~~~~\frac{df(x)}{dx} = knx^{n-1}, \textrm{for $k$ and $n$ constants}&&\\\nonumber
f(x) &=ke^{nx}~~~~\frac{df(x)}{dx}= kne^{nx}&&\\\nonumber
f(x) &=ke^{g(x)}~~~~\frac{df(x)}{dx}= ke^{g(x)}\frac{dg(x)}{dx}&&\\\nonumber
f(x) &= \ln(x)~~~~\frac{df(x)}{dx}= \frac{1}{x}&&\\\nonumber
f(x) &= \sin(x)~~~~\frac{df(x)}{dx}= \cos(x)&&\\\nonumber
f(x) &= \cos(x)~~~~\frac{df(x)}{dx}= -\sin(x)&&\\\nonumber
f(x) &= a^x~~~~\frac{df(x)}{dx}=a^x\,\ln(a), \textrm{with $a\neq1$}&&\\\nonumber
f(x) &= \tan^{-1}(x)~~~~\frac{df(x)}{dx}= \frac{1}{1+ x^2}&&\\\nonumber
\end{flalign}
\endgroup
The derivative is a linear operator, thus the following properties apply:
\begin{equation}
\frac{d(k\,f(x))}{dx} = k\,\frac{df(x)}{dx}, \textrm{with $k$ constant}
\end{equation}
and
\begin{equation}
\frac{d(f(x) + g(x))}{dx} = \frac{df(x)}{dx} + \frac{dg(x)}{dx}
\end{equation}
In addition, the differentiation of the product of two functions $f(x)$ and $g(x)$ is:
\begin{equation}
\frac{d(f(x)\times g(x))}{dx} = f(x)\frac{dg(x)}{dx} + g(x)\frac{d(f(x))}{dx}
\end{equation}
The differentiation of the quotient of two functions is:
\begin{equation}
\frac{d(f(x)/g(x))}{dx} = \frac{g(x)\frac{df(x)}{dx} - f(x)\frac{dg(x)}{dx}}{g^2(x)}
\end{equation}
Finally, the differentiation of a function of a function follows the rule in the sequel:
\begin{equation}
\frac{df(g(x))}{dx} = \frac{df(g)}{dg}\frac{g(x)}{dx}
\end{equation}
If $n$ successive derivatives must be performed, then the differential is written as:
\begin{equation}
\frac{df^n(x)}{dx^n}\equiv\frac{d^{n-1}}{x^{n-1}}\left(\frac{df(x)}{dx}\right)
\end{equation}
Geometrically, the differential represents the slope of a funtion in a given point. This, in turn, allows us to find the turning points of a function, namely its maximum and minimum, since at those points the slope will be zero. A function $f(x)$ is increasing (namely, $f(x)$ increases as $x$ increases) if the slope of the tangent in $x$ is positive. Conversely, a function $f(x)$ is decreasing (namely, $f(x)$ decreases as $x$ increases) if the slope of the tangent in $x$ is negative.

A shortand for derivative is the following:
\[\frac{df(x)}{dx}\equiv f'(x), \frac{df^2(x)}{dx^2}\equiv f''(x),\ldots\]
When $x(t)$ is a funtion of time, dot notation is used:
\[\frac{dx(t)}{dt}\equiv \dot{x}, \frac{dx^2(t)}{dt^2}\equiv \ddot{x},\ldots\]
In the case of multivariate function, when we are interested in the variation with respect just one variable, we use partial differentials to indicate this. Namely, for $f(x, t)$, the differential is either $\frac{\partial f}{\partial x}$ keeping $t$ fixed, or $\frac{\partial f}{\partial t}$ keeping $x$ fixed.